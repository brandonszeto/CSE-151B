\textbf{Number of learnable parameters and feature map dimensions} \\
Suppose we had the following architecture:
\begin{verbatim}
        inputs -> conv1 -> conv2 -> maxpool -> conv3 -> fc1 -> fc2 (outputs)
    \end{verbatim}
All convolutions have stride 1 and no zero padding unless mentioned. conv1 has a 5x5 kernel with 5 output channels. conv2 has a 7x7 kernel with 10 output channels. maxpool has 3x3 kernel with stride 2. conv3 has single output channel with a 13x13 kernel and stride=10. fc2 has 512 input units. You are supposed to predict 10 class outputs.\\
If ReLU is the activation function between each layer and the inputs are $[256 \times 256]$ RGB images, what are:
\begin{enumerate}
	\item \textbf{conv1}, \textbf{conv2}, \textbf{conv3} filter shape
	\item \textbf{conv1} output feature map shape
	\item \textbf{conv2} output feature map shape
	\item \textbf{maxpool} output feature map shape
	\item \textbf{conv3} output feature map shape
	\item Number of input and output units of \textbf{fc1} and \textbf{fc2} layer
\end{enumerate}
Show your work with calculations (6 pts)

\begin{tcolorbox}[title=Solution]
	\begin{enumerate}
		\item \textbf{conv1} has filter shape ($5 \times 5 \times 3 \times 5$). As specified above, \textbf{conv1} has a $5 \times 5$ kernel with $5$ output channels. Since an RGB image has 3 input channels, we add an additional dimension of $3$. \\
		      \textbf{conv2} has filter shape ($7 \times 7 \times 5 \times 10$). As specified above, \textbf{conv2} has a $7 \times 7$ kernel with $10$ output channels. Since \textbf{conv1} has an output of $5$ channels, \textbf{conv2} has an input of $5$ channels.\\
		      \textbf{conv3} has filter shape ($13 \times 13 \times 10 \times 1$). As specified above, \textbf{conv2} has a $13 \times 13$ kernel with a single output channel. Since \textbf{conv2} has an output of $10$ channels, \textbf{conv3} has an input of $10$ channels.
		\item \textbf{conv1} output feature map shape. Using $\textbf{Output Size} = \frac{\mathbf{W} - \mathbf{F} + \mathbf{2P}}{\mathbf{S}} + 1$, we have $\frac{256 - 5}{1} + 1 = 252$ in each dimension of the image. This results in shape $252 \times 252$ for each of $5$ output channels, resulting in overall shape ($252 \times 252 \times 5$).
		\item \textbf{conv2} output feature map shape. Using $\textbf{Output Size} = \frac{\mathbf{W} - \mathbf{F} + \mathbf{2P}}{\mathbf{S}} + 1$, we have $\frac{252 - 7}{1} + 1 = 246$ in each dimension of the image. This results in shape $246 \times 246$ for each of $10$ output channels, resulting in overall shape ($246 \times 246 \times 10$).
		\item \textbf{maxpool} output feature map shape. Using $\textbf{Output Size} = \frac{\mathbf{W} - \mathbf{F} + \mathbf{2P}}{\mathbf{S}} + 1$, we have $\frac{246 - 3}{2} + 1 = 122$ in each dimension of the image. This results in shape $122 \times 122$ for each of $10$ output channels, resulting in overall shape ($122 \times 122 \times 10$).
		\item \textbf{conv3} output feature map shape. Using $\textbf{Output Size} = \frac{\mathbf{W} - \mathbf{F} + \mathbf{2P}}{\mathbf{S}} + 1$, we have $\frac{122 - 13}{10} + 1 = 11$ in each dimension of the image. This results in shape $11 \times 11$ for a single output channel, resulting in overall shape ($11 \times 11 \times 1$).
		\item Number of input and output units of \textbf{fc1} and \textbf{fc2} layer. Since we have an output shape $11 \times 11 \times 1$ from \textbf{conv3}, we have an input size of $121$ to \textbf{fc1}. \textbf{fc1} can have an arbitrary $n$ number of output units. As stated in the problem, \textbf{fc2} expects $512$ input units, and must predict $10$ classes. Thus it must have $10$ outupt units.
	\end{enumerate}
\end{tcolorbox}
