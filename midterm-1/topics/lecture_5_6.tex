\section{Convnets}

\textbf{Main Idea}: Multiple filters are run over an image to learn different features (e.g. weights are shared). Example: $224 \times 224 \times 3 * (5 \times 5 \times 3) \rightarrow 217 \times 217 \times 1$ 

$$W_{\text{out}} = \frac{W_{\text{in}} - F - 2P}{S} + 1$$

\textbf{Properties}

\textbf{Locality}: nearby pixels correlate the most with nearby pixels, not pixels far away

\textbf{Stationary Statistics}: the statistics of pixels are relatively uniform across the image

\textbf{Translation Invariance}: identity of an object doesn’t depend on its location in the image. This is a result of conv + pool layers. A conv layer detects an object and pooling layers preserves the most prominent features in different regions and discards the rest, making it less sensitive to shifts.

\textbf{Compositionality}: objects are made of parts

\textit{Receptive fields get larger deeper into the network.}

\textbf{Pool Layer}
Reduces spatial size and number of parameters to prevent overfitting. Increases receptive field.
